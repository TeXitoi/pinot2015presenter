\documentclass{roadef}

\begin{document}

\title{Présenter une~liste~de~dates de~manière~lisible: complexité~et~algorithmes}

% Le titre court
\def\shorttitle{Présenter une~liste~de~dates de~manière~lisible}

% Les auteurs et leur numéro d'affiliation
\author{Guillaume Pinot\inst{1}}

% Les affiliations (par ordre croissant des numéros d'affiliation) séparées par \and
\institute{
  Canal TP, 20 rue Hector Malot, 75012 Paris\\
  \email{guillaume.pinot@canaltp.fr}
}

% Création de la page de titre
\maketitle
\thispagestyle{empty}

% Les mots-clés
\keywords{recherche opérationnelle, optimisation.}


\section{Introduction}

Canal TP

navitia et navitia.io

Dans navitia, nous fournissons les dates auxquelles circulent les
véhicules.  Or, une liste de dates est peu lisible par un humain. Nous
avons donc besoin de décrire sous forme condensé cette liste de dates.

\section{Le problème}

\subsection{Description du problème}

En entrée du problème, nous avons donc une liste de dates.  En sortie,
nous voulons écrire une phrase du genre \emph{du Lundi au Vendredi du
  27 avril au 31 mai 2015 sauf les 1, 8, 14 et 25 mai et le 30 mai}.

Notre résultat à notre problème doit donc décrire notre liste de dates
avec:
\begin{itemize}
\item un rythme hebdomadaire (ex: du Lundi au Vendredi);
\item une période de validité (ex: du 27 avril au 31 mai);
\item une liste de date exclues (ex: sauf les 1, 8, 14 et 25 mai);
\item une liste de dates incluses (ex: et le 6 mai).
\end{itemize}

Pour que notre description soit le plus simple possible, il faut
minimiser le nombre d'exceptions (les dates excluses et incluses).

Nous pouvons facilement transformer la liste de dates en succession de
rythmes hebdomadaire. Pour notre exemple, cela donnerait:
\begin{itemize}
\item Semaines du 27 avril et du 4 mai: Du Lundi au Jeudi;
\item Semaine du 11 mai: Du Lundi au Vendredi sauf Jeudi;
\item Semaine du 18 mai: Du Lundi au Vendredi;
\item Semaine du 25 mai: Du Mardi au Samedi.
\end{itemize}

Pour un rythme hebdomadaire donnée, il est facile de généré notre
résultat.  Le problème est donc de trouver le rythme hebdomadaire
journalier tel que le nombre d'exceptions soit minimal.

\subsection{Modélisation mathématiques}

rythme hebdomadaire -> chaînes de 0 et 1

distance de Hamming

modèle

\section{Comparaison avec l'existant et complexité}

The closest string problem, NP-hard, \cite{lanctot2003distinguishing}

réduc min max -> min sum?

notre cas avec alphabet binaire et chaines de 7 caractères.

\section{Algorithme}

Calcul de la matrice de distance, voir Tableau \ref{matrice-distances}.

\begin{table}[!ht]
    \begin{center}
        \begin{tabular}{|c|cccccccc|}
            \hline
            & 000 & 001 & 010 & 011 & 100 & 101 & 110 & 111\\
            \hline
            000 & 0 & 1 &   1 &   2 &   1 &   2 &   2 &   3\\
            001 & 1 & 0 &   2 &   1 &   2 &   1 &   3 &   2\\
            010 & 1 & 2 &   0 &   1 &   2 &   3 &   1 &   2\\
            011 & 2 & 1 &   1 &   0 &   3 &   2 &   2 &   1\\
            100 & 1 & 2 &   2 &   3 &   0 &   1 &   1 &   2\\
            101 & 2 & 1 &   3 &   2 &   1 &   0 &   2 &   1\\
            110 & 2 & 3 &   1 &   2 &   1 &   2 &   0 &   1\\
            111 & 3 & 2 &   2 &   1 &   2 &   1 &   1 &   0\\
            \hline
        \end{tabular}
        \caption{Matrice de distance pour les chaînes de 3 caractères}
    \label{matrice-distances}
    \end{center}
\end{table}

Création du vecteur de poid, évaluation d'une solution, brut force sur
les patterns.

exemple avec 011 101 110 -> 111

\section{Conclusions et perspectives}

conclusions

perspectives: Plusieurs patterns, Gestion des semaines partielles aux
bords.

\bibliographystyle{plain}
\bibliography{pinot2015presenter}

\end{document}
