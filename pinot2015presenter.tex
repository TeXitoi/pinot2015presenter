\documentclass{roadef}

\def\S{\mathcal{S}}

\begin{document}

\title{Présenter une~liste~de~dates de~manière~lisible: complexité~et~algorithme}

% Le titre court
\def\shorttitle{Présenter une~liste~de~dates de~manière~lisible}

% Les auteurs et leur numéro d'affiliation
\author{Guillaume Pinot}

% Les affiliations (par ordre croissant des numéros d'affiliation) séparées par \and
\institute{
  Canal TP, 20 rue Hector Malot, 75012 Paris\\
  \email{guillaume.pinot@canaltp.fr}
}

% Création de la page de titre
\maketitle
\thispagestyle{empty}

% Les mots-clés
\keywords{complexité, algorithme, \emph{closest string problem}.}


\section{Introduction}

Canal TP est un éditeur logiciel fournissant l'expertise et les
solutions liées à l'information voyageurs y compris dans les
transports en commun. Pour répondre à cette problématique, elle
propose le moteur navitia disponible en logiciel libre ainsi que via
le \emph{web service} \texttt{http://navitia.io/}. Ce moteur propose
entre autre un calculateur d'itinéraire et un accès aux différentes
données de transport.

Dans navitia, les dates auxquelles circulent les véhicules de
transport en commun sont fournies.  Or, une liste de dates est peu
lisible par un humain. Il est donc nécessaire de décrire sous forme
condensée cette liste de dates.

\section{Le problème}

\subsection{Description du problème}

Une liste de dates constitue donc l'entrée de notre problème. La
solution est une phrase, la plus courte possible, du genre \emph{du
  Lundi au Vendredi du 27 avril au 31 mai 2015 sauf les 1, 8, 14 et 25
  mai, avec en plus le 30 mai}.

Le résultat de ce problème doit donc décrire la liste de dates avec:
\begin{itemize}
\item un rythme hebdomadaire (exemple: du Lundi au Vendredi);
\item une période de validité (exemple: du 27 avril au 31 mai);
\item une liste de date exclues (exemple: sauf les 1, 8, 14 et 25 mai);
\item une liste de dates incluses (exemple: avec en plus le 30 mai).
\end{itemize}

Pour que la description soit la plus simple possible, il faut
minimiser le nombre d'exceptions (les dates exclues et incluses).

Il est facile de transformer la liste de dates en succession de
rythmes hebdomadaires. Pour notre exemple, cela donne:
\begin{itemize}
\item Semaines du 27 avril et du 4 mai: Du Lundi au Jeudi;
\item Semaine du 11 mai: Du Lundi au Vendredi sauf Jeudi;
\item Semaine du 18 mai: Du Lundi au Vendredi;
\item Semaine du 25 mai: Du Mardi au Samedi.
\end{itemize}

Pour un rythme hebdomadaire donné, il est facile de généré notre
résultat.  Le problème est donc de trouver le rythme hebdomadaire
journalier tel que le nombre d'exceptions soit minimal.

\subsection{Modélisation mathématiques}

Le rythme hebdomadaire peut être modélisé par une chaîne de 7~bits: un
caractère par jour de la semaine, 0 pour un jour inactif, 1 pour un
jour actif. La chaîne correspondant au rythme hebdomadaire «~du Lundi
au Vendredi~» est donc «~1111100~».

Les données du problème peuvent alors être modélisées comme une liste
de chaînes binaires de longueur 7 que l'on appellera $\S$.

Les données de l'exemple sont donc
\begin{math}
  \S = 1111000, 1111000, 1110100, 1111100, 0111110
\end{math}.

Pour décrire un rythme hebdomadaire $s_1$ en fonction d'un autre
$s_2$, le nombre d'exception correspond au nombre de substitutions de
caractère nécessaire pour aller de $s_1$ à $s_2$. Cette distance est
appelé la distance de Hamming \cite{hamming1950error}. On la notera
$d(s_1, s_2)$.

Pour résoudre notre problème, nous devons donc trouver la chaîne $s$
qui minimise la somme des distances avec $\S$. Ce qui nous donne la
formalisation suivante:

Soit $\S = s_1, s_2, \ldots, s_n$, $n$ chaînes binaires de
longueur 7. Soit
\begin{math}
  d^\Sigma_{s_i} = \sum_{s_j\in\S} d(s_i, s_j)
\end{math}.
Le meilleur rythme hebdomadaire est représenté par la chaîne $s$ tel
que $d^\Sigma_s$ soit minimale.

\section{Comparaison avec l'existant et complexité}

Le \emph{closest string problem} est un problème très proche.  Il se
définit ainsi: soit $\S = s_1, s_2, \ldots, s_n$, $n$ chaînes sur
l'alphabet $\Sigma$ de longueur $m$.  Soit $d^{\max}_{s_i} =
\max_{s_j\in\S} d(s_i, s_j)$. Le \emph{closest string problem} a pour
solution la chaîne~$s$ de longueur $m$ tel que $d^{\max}_s$ soit
minimale.  Ce problème est NP-difficile
\cite{lanctot2003distinguishing} et sa complexité paramétrique est
\begin{math}
  O(nm + nd^{\max}_s(16|\Sigma|)^{d^{\max}_s})
\end{math}
\cite{ma2008more}.

Notre problème est donc le \emph{closest string problem} avec $\Sigma
= \{0, 1\}, m = 7$ et une fonction objectif différente (notre problème
minimise la somme $d^\Sigma_s$ au lieu de minimiser le maximum
$d^{\max}_s$). La complexité paramétrique du \emph{closest string
  problem} n'étant pas exponentiel sur $n$, nous pouvons espérer
trouver un algorithme polynomial pour résoudre notre problème (avec
une constante potentiellement élevée fonction de $|\Sigma|^m = 2^7 =
128$).

\section{Algorithme}

Pour résoudre notre problème, nous proposons d'énumérer toutes les
chaînes candidates, et d'en évaluer chaque coût.  La chaîne minimisant
le coût est finalement retourné. Nous avons $|\Sigma|^m = 2^7 = 128$
chaînes possibles. Évaluer une solution a pour complexité $O(mn) =
O(7n) = O(n)$. Ainsi, la complexité globale de notre algorithme est
$O(mn\cdot|\Sigma|^m) = O(7n\cdot128) = O(n)$. En pratique, le temps
de résolution est négligeable.

\section{Conclusions et perspectives}

Un petit problème pratique à première vue simple est présenté dans cet
article. Ce problème se révèle finalement être très proche du
\emph{closest string problem} qui est NP-difficile. Comme la taille de
certains paramètres est fixée, le problème reste malgré tout
polynômial.

Ce problème pourrait être légèrement modifié pour en améliorer le
résultat. Les semaines partielles sur les bords de la période de
validité pourraient être prises en compte pour minimiser les
exceptions. Plusieurs rythmes hebdomadaires pourraient être retournés
pour condenser encore plus la phrase produite.

\bibliographystyle{plain}
\bibliography{pinot2015presenter}

\end{document}
